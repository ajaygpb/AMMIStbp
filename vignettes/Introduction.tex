\documentclass[]{article}
\usepackage{lmodern}
\usepackage{amssymb,amsmath}
\usepackage{ifxetex,ifluatex}
\usepackage{fixltx2e} % provides \textsubscript
\ifnum 0\ifxetex 1\fi\ifluatex 1\fi=0 % if pdftex
  \usepackage[T1]{fontenc}
  \usepackage[utf8]{inputenc}
\else % if luatex or xelatex
  \ifxetex
    \usepackage{mathspec}
  \else
    \usepackage{fontspec}
  \fi
  \defaultfontfeatures{Ligatures=TeX,Scale=MatchLowercase}
\fi
% use upquote if available, for straight quotes in verbatim environments
\IfFileExists{upquote.sty}{\usepackage{upquote}}{}
% use microtype if available
\IfFileExists{microtype.sty}{%
\usepackage{microtype}
\UseMicrotypeSet[protrusion]{basicmath} % disable protrusion for tt fonts
}{}
\usepackage[margin=1in]{geometry}
\usepackage{hyperref}
\hypersetup{unicode=true,
            pdftitle={The AMMIStbP package: A brief introduction},
            pdfauthor={Ajay B. C.1, J. Aravind2 and R. Abdul Fiyaz3},
            pdfborder={0 0 0},
            breaklinks=true}
\urlstyle{same}  % don't use monospace font for urls
\usepackage{color}
\usepackage{fancyvrb}
\newcommand{\VerbBar}{|}
\newcommand{\VERB}{\Verb[commandchars=\\\{\}]}
\DefineVerbatimEnvironment{Highlighting}{Verbatim}{commandchars=\\\{\}}
% Add ',fontsize=\small' for more characters per line
\usepackage{framed}
\definecolor{shadecolor}{RGB}{248,248,248}
\newenvironment{Shaded}{\begin{snugshade}}{\end{snugshade}}
\newcommand{\AlertTok}[1]{\textcolor[rgb]{0.94,0.16,0.16}{#1}}
\newcommand{\AnnotationTok}[1]{\textcolor[rgb]{0.56,0.35,0.01}{\textbf{\textit{#1}}}}
\newcommand{\AttributeTok}[1]{\textcolor[rgb]{0.77,0.63,0.00}{#1}}
\newcommand{\BaseNTok}[1]{\textcolor[rgb]{0.00,0.00,0.81}{#1}}
\newcommand{\BuiltInTok}[1]{#1}
\newcommand{\CharTok}[1]{\textcolor[rgb]{0.31,0.60,0.02}{#1}}
\newcommand{\CommentTok}[1]{\textcolor[rgb]{0.56,0.35,0.01}{\textit{#1}}}
\newcommand{\CommentVarTok}[1]{\textcolor[rgb]{0.56,0.35,0.01}{\textbf{\textit{#1}}}}
\newcommand{\ConstantTok}[1]{\textcolor[rgb]{0.00,0.00,0.00}{#1}}
\newcommand{\ControlFlowTok}[1]{\textcolor[rgb]{0.13,0.29,0.53}{\textbf{#1}}}
\newcommand{\DataTypeTok}[1]{\textcolor[rgb]{0.13,0.29,0.53}{#1}}
\newcommand{\DecValTok}[1]{\textcolor[rgb]{0.00,0.00,0.81}{#1}}
\newcommand{\DocumentationTok}[1]{\textcolor[rgb]{0.56,0.35,0.01}{\textbf{\textit{#1}}}}
\newcommand{\ErrorTok}[1]{\textcolor[rgb]{0.64,0.00,0.00}{\textbf{#1}}}
\newcommand{\ExtensionTok}[1]{#1}
\newcommand{\FloatTok}[1]{\textcolor[rgb]{0.00,0.00,0.81}{#1}}
\newcommand{\FunctionTok}[1]{\textcolor[rgb]{0.00,0.00,0.00}{#1}}
\newcommand{\ImportTok}[1]{#1}
\newcommand{\InformationTok}[1]{\textcolor[rgb]{0.56,0.35,0.01}{\textbf{\textit{#1}}}}
\newcommand{\KeywordTok}[1]{\textcolor[rgb]{0.13,0.29,0.53}{\textbf{#1}}}
\newcommand{\NormalTok}[1]{#1}
\newcommand{\OperatorTok}[1]{\textcolor[rgb]{0.81,0.36,0.00}{\textbf{#1}}}
\newcommand{\OtherTok}[1]{\textcolor[rgb]{0.56,0.35,0.01}{#1}}
\newcommand{\PreprocessorTok}[1]{\textcolor[rgb]{0.56,0.35,0.01}{\textit{#1}}}
\newcommand{\RegionMarkerTok}[1]{#1}
\newcommand{\SpecialCharTok}[1]{\textcolor[rgb]{0.00,0.00,0.00}{#1}}
\newcommand{\SpecialStringTok}[1]{\textcolor[rgb]{0.31,0.60,0.02}{#1}}
\newcommand{\StringTok}[1]{\textcolor[rgb]{0.31,0.60,0.02}{#1}}
\newcommand{\VariableTok}[1]{\textcolor[rgb]{0.00,0.00,0.00}{#1}}
\newcommand{\VerbatimStringTok}[1]{\textcolor[rgb]{0.31,0.60,0.02}{#1}}
\newcommand{\WarningTok}[1]{\textcolor[rgb]{0.56,0.35,0.01}{\textbf{\textit{#1}}}}
\usepackage{longtable,booktabs}
\usepackage{graphicx,grffile}
\makeatletter
\def\maxwidth{\ifdim\Gin@nat@width>\linewidth\linewidth\else\Gin@nat@width\fi}
\def\maxheight{\ifdim\Gin@nat@height>\textheight\textheight\else\Gin@nat@height\fi}
\makeatother
% Scale images if necessary, so that they will not overflow the page
% margins by default, and it is still possible to overwrite the defaults
% using explicit options in \includegraphics[width, height, ...]{}
\setkeys{Gin}{width=\maxwidth,height=\maxheight,keepaspectratio}
\IfFileExists{parskip.sty}{%
\usepackage{parskip}
}{% else
\setlength{\parindent}{0pt}
\setlength{\parskip}{6pt plus 2pt minus 1pt}
}
\setlength{\emergencystretch}{3em}  % prevent overfull lines
\providecommand{\tightlist}{%
  \setlength{\itemsep}{0pt}\setlength{\parskip}{0pt}}
\setcounter{secnumdepth}{0}
% Redefines (sub)paragraphs to behave more like sections
\ifx\paragraph\undefined\else
\let\oldparagraph\paragraph
\renewcommand{\paragraph}[1]{\oldparagraph{#1}\mbox{}}
\fi
\ifx\subparagraph\undefined\else
\let\oldsubparagraph\subparagraph
\renewcommand{\subparagraph}[1]{\oldsubparagraph{#1}\mbox{}}
\fi

%%% Use protect on footnotes to avoid problems with footnotes in titles
\let\rmarkdownfootnote\footnote%
\def\footnote{\protect\rmarkdownfootnote}

%%% Change title format to be more compact
\usepackage{titling}

% Create subtitle command for use in maketitle
\newcommand{\subtitle}[1]{
  \posttitle{
    \begin{center}\large#1\end{center}
    }
}

\setlength{\droptitle}{-2em}

  \title{The \textbf{AMMIStbP} package: A brief introduction}
    \pretitle{\vspace{\droptitle}\centering\huge}
  \posttitle{\par}
    \author{Ajay B. C.\textsuperscript{1}, J. Aravind\textsuperscript{2} and R.
Abdul Fiyaz\textsuperscript{3}}
    \preauthor{\centering\large\emph}
  \postauthor{\par}
      \predate{\centering\large\emph}
  \postdate{\par}
    \date{2018-07-09}

\usepackage{fancyhdr}
\usepackage{wrapfig}
\pagestyle{fancy}
\fancyhead[LE,RO]{\slshape \rightmark}
\fancyhead[LO,RE]{The AMMIStbP package{:} A brief introduction}
\fancyfoot[C]{\thepage}
\usepackage{hyperref}
\hypersetup{colorlinks=true}
\hypersetup{linktoc=all}
\hypersetup{linkcolor=blue}
\usepackage{pdflscape}
\usepackage{booktabs}
\usepackage[table]{xcolor}
\newcommand{\blandscape}{\begin{landscape}}
\newcommand{\elandscape}{\end{landscape}}

\begin{document}
\maketitle

\begin{center}
1. RRS, ICAR-Directorate of Groundnut Research, Anantapur.

2. ICAR-National Bureau of Plant Genetic Resources, New Delhi.

3. ICAR-Indian Institute of Rice Research, Hyderabad.
\end{center}

\begin{center}
\vspace{6pt}
\hrule
\end{center}

\tableofcontents

The package \texttt{AMMIStbP} is \ldots{}\ldots{}\ldots{}

\hypertarget{installation}{%
\subsection{Installation}\label{installation}}

The package can be installed using the following functions:

\begin{Shaded}
\begin{Highlighting}[]
\CommentTok{# Install from CRAN}
\KeywordTok{install.packages}\NormalTok{(}\StringTok{'AMMIStbP'}\NormalTok{, }\DataTypeTok{dependencies=}\OtherTok{TRUE}\NormalTok{)}

\CommentTok{# Install development version from Github}
\NormalTok{devtools}\OperatorTok{::}\KeywordTok{install_github}\NormalTok{(}\StringTok{"ajaygpb/AMMIStbP"}\NormalTok{)}
\end{Highlighting}
\end{Shaded}

Then the package can be loaded using the function

\begin{Shaded}
\begin{Highlighting}[]
\KeywordTok{library}\NormalTok{(AMMIStbP) }\CommentTok{# change eval}
\end{Highlighting}
\end{Shaded}

\hypertarget{ammi}{%
\subsection{AMMI}\label{ammi}}

The AMMI equation

\[Y_{ij} = \mu + \alpha_{i} + \beta_{j} + \sum_{n=1}^{N}\lambda_{n}\gamma_{in}\delta_{jn} + \rho_{ij}\]

Where, \(Y_{ij}\) is the yield of \(i\)\textsuperscript{th} genotype in
\(j\)\textsuperscript{th} environment, \(\mu\) is the grand mean,
\(\alpha_{i}\) is the genotype deviation from the grand mean,
\(\beta_{j}\) is the environment deviation, \(N\) is the total number of
interaction principal components (IPCs), \(\lambda_{n}\) is the is the
singular value for IPC \(n\) and correspondingly \(\lambda_{n}^{2}\) is
its eigen value, \(\gamma_{in}\) is the eigenvector value for
\(i\)\textsuperscript{th} genotype, \(\delta_{jn}\) is the eigenvector
value for \(j\)\textsuperscript{th} environment and \(\rho_{ij}\) is the
residual.

\hypertarget{ammi-stability-parameters}{%
\subsection{AMMI stability parameters}\label{ammi-stability-parameters}}

The details about AMMI stability parameters/indices implemented in
\texttt{AMMIStbP} are described in Table 1.

\newpage
\begin{landscape}

\rowcolors{2}{gray!25}{white}
\renewcommand{\arraystretch}{1.3}

\textbf{Table 1 :} AMMI stability parameters/indices implemented in
\texttt{AMMIStbP}. \footnotesize

\begin{longtable}[]{@{}llll@{}}
\toprule
\begin{minipage}[b]{0.14\columnwidth}\raggedright
AMMI stability parameter\strut
\end{minipage} & \begin{minipage}[b]{0.18\columnwidth}\raggedright
function\strut
\end{minipage} & \begin{minipage}[b]{0.36\columnwidth}\raggedright
Details\strut
\end{minipage} & \begin{minipage}[b]{0.20\columnwidth}\raggedright
Reference\strut
\end{minipage}\tabularnewline
\midrule
\endhead
\begin{minipage}[t]{0.14\columnwidth}\raggedright
Sums of the absolute value of the IPC scores (\(SIPC\))\strut
\end{minipage} & \begin{minipage}[t]{0.18\columnwidth}\raggedright
\texttt{SIPC.AMMI}\strut
\end{minipage} & \begin{minipage}[t]{0.36\columnwidth}\raggedright
\[SIPC =
\sum_{n=1}^{N'}
\left |
\lambda_{n}^{0.5}\gamma_{in}
\right
|\]\\
\[SIPC =
\sum_{n=1}^{N'}\left
| IPC_{n}
\right |\]\strut
\end{minipage} & \begin{minipage}[t]{0.20\columnwidth}\raggedright
Sneller et al.
(\protect\hyperlink{ref-sneller_repeatability_1997}{1997})\strut
\end{minipage}\tabularnewline
\begin{minipage}[t]{0.14\columnwidth}\raggedright
Averages of the squared eigenvector values \(EV\)\strut
\end{minipage} & \begin{minipage}[t]{0.18\columnwidth}\raggedright
\texttt{EV.AMMI}\strut
\end{minipage} & \begin{minipage}[t]{0.36\columnwidth}\raggedright
\[EV =
\sum_{n=1}^{N'}\frac{\gamma_{in}^2}{N'}\]\strut
\end{minipage} & \begin{minipage}[t]{0.20\columnwidth}\raggedright
Zobel (\protect\hyperlink{ref-zobel_stress_1994}{1994})\strut
\end{minipage}\tabularnewline
\begin{minipage}[t]{0.14\columnwidth}\raggedright
Sum across environments of GEI modelled by AMMI \(AMGE\)\strut
\end{minipage} & \begin{minipage}[t]{0.18\columnwidth}\raggedright
\texttt{AMGE.AMMI}\strut
\end{minipage} & \begin{minipage}[t]{0.36\columnwidth}\raggedright
\[AMGE =
\sum_{j=1}^{E}
\sum_{n=1}^{N'}
\lambda_{n}
\gamma_{in}
\delta_{jn}\]\strut
\end{minipage} & \begin{minipage}[t]{0.20\columnwidth}\raggedright
Sneller et al.
(\protect\hyperlink{ref-sneller_repeatability_1997}{1997})\strut
\end{minipage}\tabularnewline
\begin{minipage}[t]{0.14\columnwidth}\raggedright
\(AV_{(AMGE)}\)\strut
\end{minipage} & \begin{minipage}[t]{0.18\columnwidth}\raggedright
\texttt{NA}\strut
\end{minipage} & \begin{minipage}[t]{0.36\columnwidth}\raggedright
\[AV_{AMGE}
=
\sum_{j=1}^{E}
\sum_{n=1}^{N'}
\left
|\lambda_{n}
\gamma_{in}
\delta_{jn}
\right
|\]\strut
\end{minipage} & \begin{minipage}[t]{0.20\columnwidth}\raggedright
Zali et al. (\protect\hyperlink{ref-zali_evaluation_2012}{2012})\strut
\end{minipage}\tabularnewline
\begin{minipage}[t]{0.14\columnwidth}\raggedright
\(D_{a}\)\strut
\end{minipage} & \begin{minipage}[t]{0.18\columnwidth}\raggedright
\texttt{NA}\strut
\end{minipage} & \begin{minipage}[t]{0.36\columnwidth}\raggedright
The unsquared Euclidean distance from the origin of significant IPC axes
(D) in the AMMI model.\\
~\\
\[D_{a} =
\sqrt{\sum_{n=1}^{N'}(\lambda_{n}\gamma_{in})^2}\]\strut
\end{minipage} & \begin{minipage}[t]{0.20\columnwidth}\raggedright
Annicchiarico
(\protect\hyperlink{ref-annicchiarico_joint_1997}{1997})\strut
\end{minipage}\tabularnewline
\begin{minipage}[t]{0.14\columnwidth}\raggedright
AMMI statistic coefficient or AMMI distance or AMMI stability index
\(D_{z}\)\strut
\end{minipage} & \begin{minipage}[t]{0.18\columnwidth}\raggedright
\texttt{NA}\strut
\end{minipage} & \begin{minipage}[t]{0.36\columnwidth}\raggedright
The distance of IPC point with origin in space. (AMMI stability index)\\
~\\
\[D_{z} =
\sqrt{\sum_{n=1}^{N'}\gamma_{in}^{2}}\]\strut
\end{minipage} & \begin{minipage}[t]{0.20\columnwidth}\raggedright
Zhang et al. (\protect\hyperlink{ref-zhang_analysis_1998}{1998})\strut
\end{minipage}\tabularnewline
\begin{minipage}[t]{0.14\columnwidth}\raggedright
AMMI stability value (ASV)\strut
\end{minipage} & \begin{minipage}[t]{0.18\columnwidth}\raggedright
\texttt{agricolae::index.AMMI}\strut
\end{minipage} & \begin{minipage}[t]{0.36\columnwidth}\raggedright
Distance from the coordinate point to the origin in a two dimensional
scattergram generated by plotting of IPC1 score against IPC2 score.\\
~\\
\[ASV =
\sqrt{\left
(\frac{SSIPC_{1}}{SSIPC_{2}}\times
PC_{1}
\right )^2
+ \left
(PC_{2}
\right
)^2} \]\strut
\end{minipage} & \begin{minipage}[t]{0.20\columnwidth}\raggedright
Purchase (\protect\hyperlink{ref-purchase_parametric_1997}{1997});
Purchase et al. (\protect\hyperlink{ref-purchase_use_1999}{1999});
Purchase et al.
(\protect\hyperlink{ref-purchase_genotype_2000}{2000})\strut
\end{minipage}\tabularnewline
\begin{minipage}[t]{0.14\columnwidth}\raggedright
Modified AMMI stability value (ASV)\strut
\end{minipage} & \begin{minipage}[t]{0.18\columnwidth}\raggedright
\texttt{MASV.AMMI}\strut
\end{minipage} & \begin{minipage}[t]{0.36\columnwidth}\raggedright
\[MASV =
\sqrt{\sum_{n=1}^{N'-1}\left
(\frac{SSIPC_{n}}{SSIPC_{n+1}}
\times
PC_{n}
\right )^2
+ \left
(PC_{N'}
\right
)^2} \]\strut
\end{minipage} & \begin{minipage}[t]{0.20\columnwidth}\raggedright
Zali et al. (\protect\hyperlink{ref-zali_evaluation_2012}{2012})\strut
\end{minipage}\tabularnewline
\begin{minipage}[t]{0.14\columnwidth}\raggedright
Absolute value of the relative contribution IPCs to the interaction
\(Za\)\strut
\end{minipage} & \begin{minipage}[t]{0.18\columnwidth}\raggedright
\texttt{NA}\strut
\end{minipage} & \begin{minipage}[t]{0.36\columnwidth}\raggedright
\[Za =
\sum_{i=1}^{N'}\left
|
\theta_{n}\gamma_{in}
\right |\]\strut
\end{minipage} & \begin{minipage}[t]{0.20\columnwidth}\raggedright
Zali et al. (\protect\hyperlink{ref-zali_evaluation_2012}{2012})\strut
\end{minipage}\tabularnewline
\begin{minipage}[t]{0.14\columnwidth}\raggedright
Stability measure based on fitted AMMI model \(FA\)\strut
\end{minipage} & \begin{minipage}[t]{0.18\columnwidth}\raggedright
\texttt{NA}\strut
\end{minipage} & \begin{minipage}[t]{0.36\columnwidth}\raggedright
\[FA =
\sum_{n=1}^{N'}\lambda_{n}^{2}\gamma_{in}^{2}\]\strut
\end{minipage} & \begin{minipage}[t]{0.20\columnwidth}\raggedright
Raju (\protect\hyperlink{ref-raju_study_2002}{2002}); Zali et al.
(\protect\hyperlink{ref-zali_evaluation_2012}{2012})\strut
\end{minipage}\tabularnewline
\begin{minipage}[t]{0.14\columnwidth}\raggedright
\(FP\)\strut
\end{minipage} & \begin{minipage}[t]{0.18\columnwidth}\raggedright
\texttt{NA}\strut
\end{minipage} & \begin{minipage}[t]{0.36\columnwidth}\raggedright
Equivalent to \(FA\), when only the first IPC axis is considered for
computation.\\
~\\
\[FP =
\lambda_{1}^{2}\gamma_{i1}^{2}\]\\
~\\
As \(\lambda_{1}^{2}\) will be same for all the genotypes, the absolute
value of \(gamma_{i1}\) alone is sufficient for comparison. So this is
also equivalent to the comparison based on biplot with first IPC
axis.\strut
\end{minipage} & \begin{minipage}[t]{0.20\columnwidth}\raggedright
Raju (\protect\hyperlink{ref-raju_study_2002}{2002}); Zali et al.
(\protect\hyperlink{ref-zali_evaluation_2012}{2012})\strut
\end{minipage}\tabularnewline
\begin{minipage}[t]{0.14\columnwidth}\raggedright
\(B\)\strut
\end{minipage} & \begin{minipage}[t]{0.18\columnwidth}\raggedright
\texttt{NA}\strut
\end{minipage} & \begin{minipage}[t]{0.36\columnwidth}\raggedright
Equivalent to \(FA\), when the first two IPC axes are considered for
computation.\\
~\\
\[B =
\sum_{n=1}^{2}\lambda_{n}^{2}\gamma_{in}^{2}\]\\
~\\
Stability comparisons based on this measure will be equivalent to the
comparisons based on biplot with first two IPC axes.\strut
\end{minipage} & \begin{minipage}[t]{0.20\columnwidth}\raggedright
Raju (\protect\hyperlink{ref-raju_study_2002}{2002}); Zali et al.
(\protect\hyperlink{ref-zali_evaluation_2012}{2012})\strut
\end{minipage}\tabularnewline
\begin{minipage}[t]{0.14\columnwidth}\raggedright
\(W_{(AMMI)}\)\strut
\end{minipage} & \begin{minipage}[t]{0.18\columnwidth}\raggedright
\texttt{NA}\strut
\end{minipage} & \begin{minipage}[t]{0.36\columnwidth}\raggedright
Equivalent to \(FA\), when all the IPC axes in the AMMI model are
considered for computation.\\
~\\
\[W_{AMMI}
=
\sum_{n=1}^{N}\lambda_{n}^{2}\gamma_{in}^{2}\]\\
~\\
Equivalent to Wricke's ecovalence.\strut
\end{minipage} & \begin{minipage}[t]{0.20\columnwidth}\raggedright
Wricke (\protect\hyperlink{ref-wricke_method_1962}{1962}); Raju
(\protect\hyperlink{ref-raju_study_2002}{2002}); Zali et al.
(\protect\hyperlink{ref-zali_evaluation_2012}{2012})\strut
\end{minipage}\tabularnewline
\begin{minipage}[t]{0.14\columnwidth}\raggedright
AMMI Stability Index (\(ASI\))\strut
\end{minipage} & \begin{minipage}[t]{0.18\columnwidth}\raggedright
\texttt{NA}\strut
\end{minipage} & \begin{minipage}[t]{0.36\columnwidth}\raggedright
\[ASI =
\sqrt{\left
[
PC_{1}^{2}
\times
\theta_{1}^{2}
\right
]+\left [
PC_{2}^{2}
\times
\theta_{2}^{2}
\right
]}\]\strut
\end{minipage} & \begin{minipage}[t]{0.20\columnwidth}\raggedright
Jambhulkar et al. (\protect\hyperlink{ref-jambhulkar_ammi_2014}{2014});
Jambhulkar et al.
(\protect\hyperlink{ref-jambhulkar_genotype_2015}{2015}); Jambhulkar et
al. (\protect\hyperlink{ref-jambhulkar_stability_2017}{2017})\strut
\end{minipage}\tabularnewline
\bottomrule
\end{longtable}

Where, \(Y_{ij}\) is the yield of \(i\)\textsuperscript{th} genotype in
\(j\)\textsuperscript{th} environment, \(\mu\) is the grand mean,
\(\alpha_{i}\) is the genotype deviation from the grand mean,
\(\beta_{j}\) is the environment deviation, \(N\) is the total number of
interaction principal components (IPCs), \(N'\) is the number of
significant IPCAs (number of IPC that were retained in the AMMI model
via F tests), \(\lambda_{n}\) is the is the singular value for IPC \(n\)
and correspondingly \(\lambda_{n}^{2}\) is its eigen value,
\(\gamma_{in}\) is the eigenvector value for \(i\)\textsuperscript{th}
genotype, \(\delta_{jn}\) is the eigenvector value for
\(j\)\textsuperscript{th} environment and \(\rho_{ij}\) is the residual.

\(SSIPC_{1}\), \(SSIPC_{2}\), \(\cdots\), \(SSIPC_{n}\) are the sum of
squares of the 1\textsuperscript{st}, 2\textsuperscript{nd}, \ldots{},
and \(n\)\textsuperscript{th} IPC. \(PC_{1}\), \(PC_{2}\), \(\cdots\),
\(PC_{n}\) are the scores of 1\textsuperscript{st},
2\textsuperscript{nd}, \ldots{}, and \(n\)\textsuperscript{th} IPC.

\(\theta_{n}\) is the percentage IPC sum of squares (IPCSS) explained by
the interaction effect.

\(E\) is the number of environments.

\end{landscape}

\newpage

\hypertarget{yield-stability-index}{%
\subsection{Yield Stability Index}\label{yield-stability-index}}

The most stable genotype need not necessarily be the high yielding
genotype. As we need to select the the most stable and highest yielding
genotypes, Yield stability index (\(YSI\)) was proposed (Farshadfar et
al. (\protect\hyperlink{ref-farshadfar_ammi_2011}{2011}), Jambhulkar et
al. (\protect\hyperlink{ref-jambhulkar_stability_2017}{2017})). \(YSI\)
is a simultaneous selection index for yield and yield stability which is
computed by summation of the ranks of the stability index/parameter and
the ranks of the mean yields. \(YSI\) is computed for all the stability
parameters/indices implemented in this package.

\[YSI = R_{SP} + R_{Y}\]

Where, \(R_{SP}\) is the stability parameter/index rank of the genotype
and R\_\{Y\} is the mean yield rank of the genotype.

\hypertarget{session-info}{%
\subsection{Session Info}\label{session-info}}

\begin{Shaded}
\begin{Highlighting}[]
\KeywordTok{sessionInfo}\NormalTok{()}
\end{Highlighting}
\end{Shaded}

\begin{verbatim}
R version 3.5.1 (2018-07-02)
Platform: x86_64-w64-mingw32/x64 (64-bit)
Running under: Windows >= 8 x64 (build 9200)

Matrix products: default

locale:
[1] LC_COLLATE=English_India.1252  LC_CTYPE=English_India.1252   
[3] LC_MONETARY=English_India.1252 LC_NUMERIC=C                  
[5] LC_TIME=English_India.1252    

attached base packages:
[1] stats     graphics  grDevices utils     datasets  methods   base     

other attached packages:
[1] readxl_1.1.0  stringi_1.1.7

loaded via a namespace (and not attached):
 [1] httr_1.3.1            jsonlite_1.5          splines_3.5.1        
 [4] gtools_3.5.0          shiny_1.0.5           Rdpack_0.7-0         
 [7] assertthat_0.2.0      expm_0.999-2          xmlparsedata_1.0.1   
[10] sp_1.2-7              highr_0.6             pander_0.6.1         
[13] cellranger_1.1.0      yaml_2.1.19           remotes_1.1.1.9000   
[16] LearnBayes_2.15.1     pillar_1.2.2          backports_1.1.2      
[19] lattice_0.20-35       goodpractice_1.0.2    digest_0.6.15        
[22] promises_1.0.1        htmltools_0.3.6       httpuv_1.4.1         
[25] Matrix_1.2-14         clisymbols_1.2.0      klaR_0.6-14          
[28] devtools_1.13.5       bibtex_0.4.2          rcmdcheck_1.2.1      
[31] questionr_0.6.2       gmodels_2.16.2        xtable_1.8-2         
[34] gdata_2.18.0          processx_3.1.0        later_0.7.1          
[37] tibble_1.4.2          combinat_0.0-8        withr_2.1.2          
[40] agricolae_1.2-8       lazyeval_0.2.1        magrittr_1.5         
[43] crayon_1.3.4          mime_0.5              deldir_0.1-15        
[46] memoise_1.1.0         evaluate_0.10.1       fs_1.2.3             
[49] nlme_3.1-137          MASS_7.3-50           xml2_1.2.0           
[52] praise_1.0.0          tools_3.5.1           hunspell_2.9         
[55] cyclocomp_1.1.0       gbRd_0.4-11           stringr_1.3.0        
[58] cluster_2.0.7-1       callr_2.0.4           rex_1.1.2            
[61] compiler_3.5.1        pkgdown_1.1.0.9000    covr_3.0.1           
[64] rlang_0.2.1           debugme_1.1.0         grid_3.5.1           
[67] rstudioapi_0.7.0-9000 miniUI_0.1.1          rmarkdown_1.10       
[70] boot_1.3-20           roxygen2_6.0.1        AlgDesign_1.1-7.3    
[73] R6_2.2.2              knitr_1.20            commonmark_1.5       
[76] rprojroot_1.3-2       spdep_0.7-7           lintr_1.0.2          
[79] desc_1.2.0            whoami_1.1.2          Rcpp_0.12.16         
[82] spData_0.2.8.3        coda_0.19-1          
\end{verbatim}

\hypertarget{references}{%
\subsection*{References}\label{references}}
\addcontentsline{toc}{subsection}{References}

\hypertarget{refs}{}
\leavevmode\hypertarget{ref-annicchiarico_joint_1997}{}%
Annicchiarico, P. (1997). Joint regression vs AMMI analysis of
genotype-environment interactions for cereals in Italy. \emph{Euphytica}
94, 53--62.
doi:\href{https://doi.org/10.1023/A:1002954824178}{10.1023/A:1002954824178}.

\leavevmode\hypertarget{ref-farshadfar_ammi_2011}{}%
Farshadfar, E., Mahmodi, N., and Yaghotipoor, A. (2011). AMMI stability
value and simultaneous estimation of yield and yield stability in bread
wheat (\emph{Triticum aestivum} L.). \emph{Australian Journal of Crop
Science} 5, 1837--1844.

\leavevmode\hypertarget{ref-jambhulkar_genotype_2015}{}%
Jambhulkar, N. N., Bose, L. K., Pande, K., and Singh, O. N. (2015).
Genotype by environment interaction and stability analysis in rice
genotypes. \emph{Ecology, Environment and Conservation} 21, 1427--1430.
Available at:
\url{http://www.envirobiotechjournals.com/article_abstract.php?aid=6346\&iid=200\&jid=3}.

\leavevmode\hypertarget{ref-jambhulkar_ammi_2014}{}%
Jambhulkar, N. N., Bose, L. K., and Singh, O. N. (2014). ``AMMI
Stability Index for Stability Analysis,'' in \emph{CRRI Newsletter,
January-March 2014}, ed. T. Mohapatra (Cuttack, Orissa: Central Rice
Research Institute), 15. Available at:
\url{http://www.crri.nic.in/CRRI_newsletter/crnl_jan_mar_14_web.pdf}.

\leavevmode\hypertarget{ref-jambhulkar_stability_2017}{}%
Jambhulkar, N., Rath, N., Bose, L., Subudhi, H., Biswajit, M., Lipi, D.,
et al. (2017). Stability analysis for grain yield in rice in
demonstrations conducted during rabi season in India. \emph{Oryza} 54,
236--240.

\leavevmode\hypertarget{ref-purchase_parametric_1997}{}%
Purchase, J. (1997). Parametric Analysis to Describe Genotype ×
Environment Interaction and Yield Stability in Winter Wheat. Available
at: \url{http://hdl.handle.net/11660/1966}.

\leavevmode\hypertarget{ref-purchase_use_1999}{}%
Purchase, J., Hatting, H., and Van Deventer, C. (1999). ``The use of the
AMMI model and AMMI stability value to describe genotype x environment
interaction and yield stability in winter wheat (\emph{Triticum
aestivum} L.),'' in \emph{Proceedings of the Tenth Regional Wheat
Workshop for Eastern, Central and Southern Africa, University of
Stellenbosch, South Africa; 14-18 September 1998}.

\leavevmode\hypertarget{ref-purchase_genotype_2000}{}%
Purchase, J. L., Hatting, H., and Deventer, C. S. van (2000). Genotype ×
environment interaction of winter wheat (Triticum aestivum L.) In South
Africa: II. Stability analysis of yield performance. \emph{South African
Journal of Plant and Soil} 17, 101--107.
doi:\href{https://doi.org/10.1080/02571862.2000.10634878}{10.1080/02571862.2000.10634878}.

\leavevmode\hypertarget{ref-raju_study_2002}{}%
Raju, B. M. K. (2002). A Study on AMMI Model and its Biplots.
\emph{Journal of the Indian Society of Agricultural Statistics} 55,
297--322.

\leavevmode\hypertarget{ref-sneller_repeatability_1997}{}%
Sneller, C. H., Kilgore-Norquest, L., and Dombek, D. (1997).
Repeatability of Yield Stability Statistics in Soybean. \emph{Crop
Science} 37, 383--390.
doi:\href{https://doi.org/10.2135/cropsci1997.0011183X003700020013x}{10.2135/cropsci1997.0011183X003700020013x}.

\leavevmode\hypertarget{ref-wricke_method_1962}{}%
Wricke, G. (1962). On a method of understanding the biological diversity
in field research. \emph{Zeitschrift für Pflanzenzüchtung} 47, 92--146.

\leavevmode\hypertarget{ref-zali_evaluation_2012}{}%
Zali, H., Farshadfar, E., Sabaghpour, S. H., and Karimizadeh, R. (2012).
Evaluation of genotype × environment interaction in chickpea using
measures of stability from AMMI model. \emph{Annals of Biological
Research} 3, 3126--3136. Available at:
\url{http://www.ijabbr.com/article_7777_620ea1a0c1fd04868f60bd23c6dda48b.pdf}.

\leavevmode\hypertarget{ref-zhang_analysis_1998}{}%
Zhang, Z., Lu, C., and Xiang, Z. (1998). Analysis of variety stability
based on AMMI model. \emph{Acta Agronomica Sinica} 24, 304--309.
Available at: \url{http://zwxb.chinacrops.org/EN/Y1998/V24/I03/304}.

\leavevmode\hypertarget{ref-zobel_stress_1994}{}%
Zobel, R. (1994). ``Stress resistance and root systems,'' in
\emph{Proceedings of the Workshop on Adaptation of Plants to Soil
Stress. 1-4 August, 1993. INTSORMIL Publication 94-2} (Institute of
Agriculture; Natural Resources, University of Nebraska-Lincoln), 80--99.


\end{document}
